%%%%%%%%%%%%%%%%%%%%%%%%%%%%%%%%%%%%%%%%%
% Important note:
% This template requires the resume.cls file to be in the same directory as the
% .tex file. The resume.cls file provides the resume style used for structuring the
% document.
%
%%%%%%%%%%%%%%%%%%%%%%%%%%%%%%%%%%%%%%%%%

%----------------------------------------------------------------------------------------
%	PACKAGES AND OTHER DOCUMENT CONFIGURATIONS
%----------------------------------------------------------------------------------------

\documentclass{resume} % Use the custom resume.cls style

\usepackage[left=1in,top=1in,right=1in,bottom=1in]{geometry} % Document margins
\usepackage{dirtytalk}
\usepackage{graphicx}
\usepackage{fancyvrb}
\usepackage{hyperref}
\hypersetup{
    breaklinks=true,
    colorlinks=true,
    linkcolor=blue,
    filecolor=blue,      
    urlcolor=blue,
    pdftitle={Overleaf Example},
    pdfpagemode=FullScreen,
    }
\newcommand{\Rlogo}{\protect\includegraphics[height=1.8ex,keepaspectratio]{Rlogo.pdf}} \newcommand{\myinput}[1] {\begin{scriptsize}
\VerbatimInput[frame=single,label=#1]{#1}
\end{scriptsize}}
\newcommand{\tab}[1]{\hspace{.2667\textwidth}\rlap{#1}}
\newcommand{\itab}[1]{\hspace{0em}\rlap{#1}}
\name{Jozef Rivest} % Your name
%\address{333 rue Laforest, Saint-Alphonse-Rodriguez, Quebec, Canada} % NOTE: À changer éventuellement !!!!! 
%\address{123 Pleasant Lane \\ City, State 12345} % Your secondary addess (optional)
\address{(+1) 450-758-0216 \\ jrivest@umich.edu} % Your phone number and email

\begin{document}

%----------------------------------------------------------------------------------------
%	EDUCATION SECTION
%----------------------------------------------------------------------------------------

\begin{rSection}{Education}
%--copy and paste this region  if you need more--
{\bf University of Michigan} \hfill {\em 2025 -- }
\\ Ph.D. Political Science

{\bf University of Montreal} \hfill {\em 2025} 
\\ M.Sc. Political Science \hfill {\em GPA: 4/4}
\\ Supervisor : Dominique Caouette 
\\ Co-Supervisor : Richard Nadeau
\\ Thesis : Security Shifts and Public Sentiment: How the Recent Shifts in East Asia Geopolitics Affect Japanese Pacifist Norms

{\bf Inter-university Consortium for Political and Social Research} (ICPSR) \hfill {\em 2024} 
\\University of Michigan \hfill 

{\bf University of the Philippines Diliman} \hfill {\em 2023} 
\\ Certificate in qualitative methods in Southeast Asia\hfill 

{\bf University of Montreal} \hfill {\em 2022}
\\ B.Sc. Hons. Political Science \hfill {\em GPA: 4.2/4.3}
\\ Honor thesis: Japan and Covid-19: Paradox and Culture

\end{rSection}

%--copy and paste this region  if you need more--

%% Interest 
\begin{rSection}{Interests}
Research Methods
\\ Comparative Politics
\\ Public Opinion
\\ Japan
\\ East Asia
\\ Southeast Asia
\end{rSection}

%----------------------------------------------------------------------------------------
%	EXPERIENCE SECTION
%----------------------------------------------------------------------------------------
\begin{rSection}{Academic Experiences}
%--copy and paste this region  if you need more--
{\bf Asie en 1000 mots}{ [Asia in 1000 words], Coordinator} \hfill {\em 2023 - }\\
Publication coordinator for l'Asie en 1000 mots, an initiative of the Center for Asian Studies of the University of Montreal (CÉTASE). The site brings together accessible analyzes on the social, political, historical, economic and cultural phenomena that shape East and Southeast Asia by specialists from academia. 

\clearpage

{\bf Research Auxiliary}
\begin{enumerate}
    \item \textbf{Analyzing Data About Attitudes Toward Poverty in Quebec} \hfill {\em 2025}
        \begin{itemize}
            \item The contract is realized with Professors Frederick Bastien (University of Montreal) and Normand Landry (TÉLUQ University). The task concern analyzing data from a survey of attitudes toward poverty in Quebec. It will lead to the creation of scales, production of a report, and production of scientific articles. 
        \end{itemize}
    \item \textbf{Contracts Realized With Pr. Vincent Arel-Bundock} \hfill {\em 2024}
        \begin{itemize}
            \item Editing Quarto documents for the Marginal Effect website\footnote{\url{https://marginaleffects.com}}
            \item Classifying scientific paper for an LLM driven research aiming to extract information from academic articles at scale
        \end{itemize}
    \item \textbf{Leadership Chair in the Teaching of Digital Social Sciences} \hfill {\em 2022 -}
        \begin{itemize}
            \item {Working on the adaptation and development of a Japanese version of Datagotchi, an educational and fun web application issuing predictions based on lifestyle habits.}
            \item Writing stories for US media about the relation between lifestyle and views on international relations, in the context of the 60th presidential election 
            \item Developing a questionnaire and a conjoint experiment aimed at evaluating the psychosocial factors that influence the acceptability of gene therapy
            \item Working on different projects and paper related to the use of AI in social sciences
            \item Wrote two chapters for a book about the numerical tools in social sciences
        \end{itemize}
    \item \textbf{Contract Realized With Pr. Laurence McFalls} \hfill {\em 2022}
        \begin{itemize}
            \item Books and documents sorting.
            \item Preparation and technical assistance for the International Research Training Group's closing conference.
        \end{itemize}
\end{enumerate}

\end{rSection}

\begin{rSection}{Professional Experiences}

%--copy and paste this region  if you need more--
{\bf Québec Solidaire}\footnote{This experience is independent of my political preferences.}  \hfill {\em 2022}
\begin{itemize}
    \item Political Advisor.
    \item Co-Coordinator of Electoral Campaign.
    \item Communications and Media Manager.
\end{itemize}

\end{rSection}

%--------------------------------------------------------------------------------
%    TEACHING EXPERIENCE
%-----------------------------------------------------------------------------------------------

\begin{rSection}{Teaching Experiences}
    \subsection*{Teaching Assistant}
        \begin{itemize}
            \item Southeast Asia - POL3401, University of Montreal, Fall 2023
            \item International Relations of Southeast Asia - POL3011, University of Montreal, Winter 2024.
        \end{itemize}
    \subsection*{Interventions}
        \begin{itemize}
            \item Artificial Intelligence in Social Sciences: Questions, Issues and Opportunities, Numeric Research Tools - POL6078, Laval University, Fall 2024
            \item An Introduction to Quarto: How to Present your Analyses Using Quarto, Quantitative Research in Political Science - POL6021, University of Montreal, Fall 2024
        \end{itemize}
        
\end{rSection}

%--------------------------------------------------------------------------------
%    PROJECTS
%-----------------------------------------------------------------------------------------------

\begin{rSection}{Publications}
%--copy and paste this region  if you need more--

\subsection*{Peer-Reviewed Articles}

\textbf{A.1} Foisy, Laurence-Olivier M., Jérémie Drouin, Camille Pelletier, \textbf{Jozef Rivest}, Hubert Cadieux, Yannick Dufresne. 2025. \say{Ain't No Party Like a GPT Party. Assessing GPT-4 Political Alignment Classification Capabilities}. \textit{Journal of Information Technology \& Politics} 

\subsection*{Peer-Reviewed Book Chapters}

{\textbf{B.2 Rivest, Jozef} et Catherine Ouellet. \say{Chapitre 2. Vers une science numérique plus transparante:
l’apport du logiciel libre et du code ouvert dans les sciences sociales [Chapter B. Toward a more transparent digital science: the contribution of free software and open source in the social sciences]}, In \textit{Outils de recherche en sciences sociales numériques [Digital research tools for the social sciences]}, Cloutier, A., Dufresne, Y., Bibeau-Gagnon, A. (Dir.). Québec: Presses de l'Université Laval. (Under Contract)} \par

{\textbf{B.1 Rivest, Jozef}, Hubert Cadieux et Laurence-Olivier M. Foisy. \say{Chapitre 9. L'Intelligence Artificielle dans les Sciences Sociales: Outils, Enjeux et Questionnement [Chapter 9. Artificial intelligence in the social sciences: tools, issues and questions]}, In \textit{Outils de recherche en sciences sociales numériques [Digital research tools for the social sciences]}, Cloutier, A., Dufresne, Y., Bibeau-Gagnon, A. (Dir.). Québec: Presses de l'Université Laval.} (Under Contract) \par

\subsection*{Conferences}

{\textbf{C.7} Bastien, Frédérick, \textbf{Jozef Rivest} et Normand Landry (TÉLUQ). \say{Aide-toi, le ciel t’aidera! » Le rôle du mérite, de l’identité partisane et des médias dans l’imputabilité d’une atténuation de la pauvreté [Help Yourself, Heaven Will Help You!” The Role of Merit, Partisan Identity, and the Media in Accountability for Poverty Alleviation]}, Société Québécoise de Science politique [Quebec's Society of Political Science], Montreal, May 2025}

{\textbf{C.6} Proulx, Étienne, \textbf{Jozef Rivest}, Camille Pelletier, Laurence-Olivier M. Foisy. \say{Cultural Dimensions of Trust and Fear: A Comparative Study of Public Attitudes Toward Artificial Intelligence in Japan and Canada} Midwest Political Science Association, {April 2025}}

{\textbf{C.5 Rivest, Jozef}. \say{Unshakable Pacifism? Japanese Pacifist and Antimilitarist Norms in the Face of East Asian Geopolitical Changes} The Northeastern Political Science Association, Boston, November 2024}

{\textbf{C.4 Rivest, Jozef}. \say{Security Shifts and Public Sentiment: How the Recent Shifts in East Asia Geopolitics Affect Japanese Pacifist Norms} Canadian Political Science Association, Montreal, 2024} \par

{\textbf{C.3 Rivest, Jozef}. \say{Un pacifisme inébranlable ? Les normes pacifiques et antimilitaristes japonaises face aux changements géopolitiques est-asiatiques [Unshakable Pacifism? Japanese Pacifist and Antimilitarist Norms in the Face of East Asian Geopolitical Changes]} Colloque Regards croisés et pluriels sur l'Indo-pacifique du CÉRIUM [Crossed and plural perspectives on the Indo-Pacific, Montreal Center for International Studies' symposium], Montreal, March 2024} \par

{\textbf{C.2 Rivest, Jozef}. \say{Vers un renouveau de la culture stratégique du Japon? [Toward a renewal of Japa's strategic culture?]} Colloque étudiant du Centre d’études et de recherche internationale de l’Université de Montréal (CÉRIUM) [Student Symposium of the Montreal Center for International Studies], Montreal, March 2023.} \par

{\textbf{C.1} Dufresne, Yannick, Axel Dery, \textbf{Jozef Rivest}, Catherine Ouellet and Etienne Gagnon. \say{What Social-Class Markers and Lifestyles Tell Us About the Left-Right Divide in Canada and Japan} Southern Political Science Association, St. Pete Beach, January 2023.}

\subsection*{Submitted}

{\textbf{S.1} Dufresne, Yannick, Hubert Cadieux, Etinne Proulx, Laurence-Olivier M. Foisy, \textbf{Jozef Rivest}, Alexandre Bouillon, and Jeremy Gilbert. \say{Open AI's GPT Model in Political Context: A New Dataset for Political Science Research}.}

\subsection*{Review and Resubmit}

{\textbf{R.1} Girard, Tyler, \textbf{Jozef Rivest}, Mathieu Turgeon, Yannick Dufresne, Takeshi Iida and Tetsuya Matsubayashi. \say{Public Support for Emerging Military Technology Development in ‘Middle’ Powers}.} \par 

\subsection*{Medias}

{\textbf{M.8 Rives, Jozef}. 2024. \say{Le Japon plongé dans l'incertitude politique [Japan plunged into political uncertainty]} \textit{Le Devoir}, 1 November 2024.}

{\textbf{M.7 Rivest, Jozef}. 2024. \say{Élections à Taïwan, une démocratie sous tension [Elections in Taiwan, a democracy under strain]} \textit{Le Devoir}, 17 janvier 2024.} \par

{\textbf{M.6 Rivest, Jozef}. 2023. \say{Pourquoi la Chine est-elle si préoccupée par Taïwan? [Why is China so concerned about Taiwan?]} \textit{Le Devoir}, 16 août 2023.} \par

{\textbf{M.5 Rivest, Jozef}. 2022. \say{Le Viêtnam : grand bénéficiaire de la rivalité sino-américaine [Vietnam: a major beneficiary of Sino-American rivalry]} \textit{Blogue sur l’Asie du Sud-Est [Southeast Asia Blog]}, 4 janvier 2023.} \par 

{\textbf{M.4 Rivest, Jozef}. 2022. \say{Le panoptique technologique au service de l’autoritarisme [Technological panopticism in the service of authoritarianism]} \textit{Blogue sur l’Asie du Sud-Est [Southeast Asia Blog]}, 30 novembre 2022.} \par

{\textbf{M.3 Rivest, Jozef}. \say{Le Parti communiste chinois ou le Parti de Xi Jinping? Retour sur le 20e Congrès national du PCC [The Chinese Communist Party of Xi Jinping's Party? A look back at the 20th National Congress of the CCP?]} \textit{La revue du CAIUM [CAIUM's Magazine]}, 1er novembre 2022.} \par

{\textbf{M.2 Rivest, Jozef}. 2022. \say{Guerre ou statu quo? Le cas de la mer de Chine méridionale [War or status quo? The case of the South China Sea]} \textit{Blogue sur l’Asie du Sud-Est [Southeast Asia blog]}, 30 novembre 2022.} \par 

{\textbf{M.1 Rivest, Jozef}. 2022. \say{Le Japon et la Covid-19 : paradoxe et culture [Japan and Covid-19: paradox and culture]}, \textit{L’Asie en 1000 mots [Asia in 1000 words]}, 3 octobre 2022.} \par

\subsection*{In Work}

\subsubsection*{Software}

{\textbf{Rivest, Jozef}. \say{factortable: Flexible and Intuitive Tables to Present Factor Analysis Results in Paper}. \textbf{R package}}

\subsubsection*{Paper}

{\textbf{Rivest, Jozef}. \say{Pacifism Under Scrutiny: Measuring Japanese Norms Amid Shifting Geopolitical Dynamics}}

{\textbf{Rivest, Jozef}, Mathieu Turgeon, Tetsuya Matsubayashi, Takeshi Iida, Yannick Dufresne. \say{Permanent or Temporary? Understanding Japanese's Preferences Toward Immigration}}

%--copy and paste this region  if you need more--
\end{rSection}

%--------------------------------------------------------------------------------
%    ACTIVITIES
%-----------------------------------------------------------------------------------------------
\begin{rSection}{Scholarships and Funding}
%--copy and paste this region  if you need more--
\begin{itemize} 
  \item Doctoral Fellowship, SSHRC. 2025-2027. 120,000CA\$
  \item Scholarship of Excellence, Nissan Scholarship. 2024. 3,000CA\$
  \item Conference grants, Center for the Study of Democratic Citizenship. 2024. 1,000CA\$
  \item Graduation grants, University of Montreal, 2024. 4,000CA\$
  \item Methods training grants, Center for the Study of Democratic Citizenship. 2024. 3,000CA\$
  \item Research Grand, Center for the Study of Democratic Citizenship. 2024. 2,500CA\$
  \item Scholarship of Excellence, Nissan Scholarship. 2023. 3,000CA\$
  \item Master's Scholarship, SSHRC. 2022. 17,500CA\$
  \item Desjardins Scholarship, Desjardins. 2021. 1,000CA\$
\end{itemize}

\end{rSection}


%% Affiliations académiques
\begin{rSection}{Affiliation} 
{\em Japanese Association of Electoral Studies} \hfill{\em 2024 - } \par
{\em Canadian Political Science Association} \hfill{\em 2024 - } \par
{\em Student Member at the Center for the Study of Democratic Citizenship} \hfill{\em 2023 - } \par
{\em Leadership Chair in the Teaching of Digital Social Sciences} \hfill{\em 2022 - }

\end{rSection}
\clearpage
%---------------k------------------------------------------------------------------------
%	SKILLS SECTION
%----------------------------------------------------------------------------------------
\begin{rSection}{Skills}
{\bf Language}
\begin{itemize}
    \item French (native)
    \item English (C1)\footnote{TOELF Internet based scores: total 111, reading 26, listening 27, speaking 30, writing 28}
    \item Japanese (A2+, JLPT N4)
\end{itemize}
{\bf Programming Languages and Frameworks}
\begin{itemize}
    \item \Rlogo 
    \item \LaTeX 
    \item Qualtrics
    \item Shiny
    \item Quarto/Markdown
\end{itemize}
\end{rSection}


\end{document}----------------------------
