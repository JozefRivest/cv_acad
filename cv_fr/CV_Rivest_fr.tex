%%%%%%%%%%%%%%%%%%%%%%%%%%%%%%%%%%%%%%%%%
% Important note:
% This template requires the resume.cls file to be in the same directory as the
% .tex file. The resume.cls file provides the resume style used for structuring the
% document.
%
%%%%%%%%%%%%%%%%%%%%%%%%%%%%%%%%%%%%%%%%%

%----------------------------------------------------------------------------------------
%	PACKAGES AND OTHER DOCUMENT CONFIGURATIONS
%----------------------------------------------------------------------------------------

\documentclass{resume} % Use the custom resume.cls style
\usepackage{CJKutf8}
\usepackage[T1]{fontenc}
\usepackage[french]{babel}

\usepackage[left=1in,top=1in,right=1in,bottom=1in]{geometry} % Document margins
\usepackage{dirtytalk}
\usepackage{graphicx}
\usepackage{fancyvrb}
\usepackage[dvipsnames]{xcolor}
\usepackage{hyperref}
\hypersetup{
    colorlinks=true,
    linkcolor=Maroon,
    filecolor=Maroon,
    urlcolor=Maroon,
    linkbordercolor=Maroon,
    breaklinks=true
}
\newcommand{\Rlogo}{\protect\includegraphics[height=1.8ex,keepaspectratio]{Rlogo.pdf}} \newcommand{\myinput}[1] {\begin{scriptsize}
\VerbatimInput[frame=single,label=#1]{#1}
\end{scriptsize}}
\newcommand{\tab}[1]{\hspace{.2667\textwidth}\rlap{#1}}
\newcommand{\itab}[1]{\hspace{0em}\rlap{#1}}
\name{Jozef Rivest} % Your name
%\address{333 rue Laforest, Saint-Alphonse-Rodriguez, Quebec, Canada} % NOTE: À changer éventuellement !!!!! 
%\address{123 Pleasant Lane \\ City, State 12345} % Your secondary addess (optional)
\address{(+1) 450-758-0216 \\ jrivest@umich.edu} % Your phone number and email

\begin{document}

%----------------------------------------------------------------------------------------
%	EDUCATION SECTION
%----------------------------------------------------------------------------------------

\begin{rSection}{Formation}
%--copy and paste this region  if you need more--
{\bf Université du Michigan} \hfill {\em 2025 -- }
\\ Ph.D. Science politique

{\bf Université de Montréal} \hfill {\em 2025} 
\\ M.Sc. Science politique \hfill {\em Cote Z : 4,3/4,3}
\\ Directeur : Dominique Caouette 
\\ Co-directeur : Richard Nadeau
\\ Mémoire : Un pacifime inébranlable? : les attitudes pacifistes et antimilitaristes japonaises au 21e siècle

{\bf Consortium interuniversitaire pour la recherche politique et sociale} (ICPSR) \hfill {\em 2024} 
\\Université du Michigan \hfill 

{\bf Université des Philippines Diliman} \hfill {\em 2023} 
\\ Certificat en méthodes qualitatives en Asie du Sud-Est\hfill 

{\bf Université de Montréal} \hfill {\em 2022}
\\ B.Sc. Hons. Science politique \hfill {\em Cote Z : 4,2/4,3}
\\ Recherche Hons. : Le Japon et la Covid-19 : Paradoxe et Culture

\end{rSection}

%--copy and paste this region  if you need more--

%% Interest 
\begin{rSection}{Intérêts}
Méthodes de recherche
\\ Politique comparée
\\ Opinion publique
\\ Japon
\\ Asie de l'Est
\\ Asie du Sud-Est
\end{rSection}

%----------------------------------------------------------------------------------------
%	EXPERIENCE SECTION
%----------------------------------------------------------------------------------------
\begin{rSection}{Expériences académiques}
%--copy and paste this region  if you need more--
{\bf Asie en 1000 mots}{ [Asia in 1000 words], Coordinateur} \hfill {\em 2023 - }\\
Coordinateur de publication pour l'Asie en 1000 mots, une initiative du Centre d'études asiatiques de l'Université de Montréal (CÉTASE). Le site rassemble des analyses accessibles sur les phénomènes sociaux, politiques, historiques, économiques et culturels qui façonnent l'Asie de l'Est et du Sud-Est par des spécialistes du milieu académique. 

\clearpage

{\bf Auxiliaire de recherche}
\begin{enumerate}
    \item \textbf{Analyse de données sur les attitudes envers la pauvreté au Québec} \hfill {\em 2025}
        \begin{itemize}
            \item Le contrat est réalisé avec les professeurs Frederick Bastien (Université de Montréal) et Normand Landry (Université TÉLUQ). La tâche concerne l'analyse de données d'une enquête sur les attitudes envers la pauvreté au Québec.
        \end{itemize}
    \item \textbf{Contrats réalisés avec le Pr. Vincent Arel-Bundock} \hfill {\em 2024}
        \begin{itemize}
          \item Édition de documents Quarto pour le \href{https://marginaleffects.com}{site web Marginal Effect}.
            \item Classification d'articles scientifiques pour une recherche pilotée par LLM visant à extraire des informations d'articles académiques à grande échelle.
        \end{itemize}
    \item \textbf{Chaire de leadership en enseignement des sciences sociales numériques} \hfill {\em 2022 -}
        \begin{itemize}
            \item {Travail sur l'adaptation et le développement d'une version japonaise de Datagotchi, une application web éducative et amusante émettant des prédictions basées sur les habitudes de vie.}
            \item Rédaction d'articles pour les médias américains sur la relation entre le mode de vie et les opinions sur les relations internationales, dans le contexte de la 60e élection présidentielle. 
            \item Développement d'un questionnaire et d'une expérience conjointe visant à évaluer les facteurs psychosociaux qui influencent l'acceptabilité de la thérapie génique.
            \item Travail sur différents projets et articles liés à l'utilisation de l'IA dans les sciences sociales.
            \item Rédaction de deux chapitres pour un livre sur les outils numériques en sciences sociales.
        \end{itemize}
    \item \textbf{Contrat réalisé avec le Pr. Laurence McFalls} \hfill {\em 2022}
        \begin{itemize}
            \item Tri de livres et de documents.
            \item Préparation et assistance technique pour la conférence de clôture du Groupe international de formation à la recherche.
        \end{itemize}
\end{enumerate}

\end{rSection}

%--------------------------------------------------------------------------------
%    TEACHING EXPERIENCE
%-----------------------------------------------------------------------------------------------

\begin{rSection}{Expériences d'enseignement}
    \subsection*{Assistant d'enseignement}
        \begin{itemize}
            \item Asie du Sud-Est - POL3401, Université de Montréal, Automne 2023.
            \item Relations internationales de l'Asie du Sud-Est - POL3011, Université de Montréal, Hiver 2024.
        \end{itemize}
    \subsection*{Interventions}
        \begin{itemize}
            \item Intelligence artificielle en sciences sociales : Questions, enjeux et opportunités, Outils de recherche numériques - POL6078, Université Laval, Automne 2024.
            \item Une introduction à Quarto : Comment présenter vos analyses en utilisant Quarto, Recherche quantitative en science politique - POL6021, Université de Montréal, Automne 2024.
        \end{itemize}
        
\end{rSection}

%--------------------------------------------------------------------------------
%    PROJECTS
%-----------------------------------------------------------------------------------------------

\begin{rSection}{Publications}
%--copy and paste this region  if you need more--

\subsection*{Articles évalués par les pairs}

{\textbf{A.2} Dufresne, Yannick, Hubert Cadieux, Etinne Proulx, Laurence-Olivier M. Foisy, \textbf{Jozef Rivest}, Alexandre Bouillon, and Jeremy Gilbert. 2025. \say{Open AI's GPT Model in Political Context: A New Dataset for Political Science Research}. \textit{Social Science Computer Review}, 0(0). \href{https://doi.org/10.1177/08944393251344865}{https://doi.org/10.1177/08944393251344865}}

{\textbf{A.1} Foisy, Laurence-Olivier M., Jérémie Drouin, Camille Pelletier, \textbf{Jozef Rivest}, Hubert Cadieux, Yannick Dufresne. 2025. \say{Ain't No Party Like a GPT Party. Assessing GPT-4 Political Alignment Classification Capabilities}. \textit{Journal of Information Technology \& Politics}, December, 1-13. \href{https://doi.org/10.1177/08944393251344865}{https://doi.org/10.1177/08944393251344865}}

\subsection*{Chapitres de livres évalués par les pairs}

{\textbf{B.2 Rivest, Jozef} et Catherine Ouellet. \say{Chapitre 2. Vers une science numérique plus transparante:
l'apport du logiciel libre et du code ouvert dans les sciences sociales}, In \textit{Outils de recherche en sciences sociales numériques}, Cloutier, A., Dufresne, Y., Bibeau-Gagnon, A. (Dir.). Québec: Presses de l'Université Laval. (Sous contrat)} \par

{\textbf{B.1 Rivest, Jozef}, Hubert Cadieux et Laurence-Olivier M. Foisy. \say{Chapitre 9. L'Intelligence Artificielle dans les Sciences Sociales: Outils, Enjeux et Questionnement}, In \textit{Outils de recherche en sciences sociales numériques}, Cloutier, A., Dufresne, Y., Bibeau-Gagnon, A. (Dir.). Québec: Presses de l'Université Laval.} (Sous contrat) \par

\subsection*{Conférences}

{\textbf{C.7} Bastien, Frédérick, \textbf{Jozef Rivest} et Normand Landry (TÉLUQ). \say{Aide-toi, le ciel t'aidera! Le rôle du mérite, de l'identité partisane et des médias dans l'imputabilité d'une atténuation de la pauvreté}, Société Québécoise de Science politique, Montréal, mai 2025.}

{\textbf{C.6} Proulx, Étienne, \textbf{Jozef Rivest}, Camille Pelletier, Laurence-Olivier M. Foisy. \say{Cultural Dimensions of Trust and Fear: A Comparative Study of Public Attitudes Toward Artificial Intelligence in Japan and Canada} Midwest Political Science Association, avril 2025.}

{\textbf{C.5 Rivest, Jozef}. \say{Unshakable Pacifism? Japanese Pacifist and Antimilitarist Norms in the Face of East Asian Geopolitical Changes} The Northeastern Political Science Association, Boston, novembre 2024.}

{\textbf{C.4 Rivest, Jozef}. \say{Security Shifts and Public Sentiment: How the Recent Shifts in East Asia Geopolitics Affect Japanese Pacifist Norms} Association canadienne de science politique, Montréal, 2024.}

{\textbf{C.3 Rivest, Jozef}. \say{Un pacifisme inébranlable ? Les normes pacifiques et antimilitaristes japonaises face aux changements géopolitiques est-asiatiques} Colloque Regards croisés et pluriels sur l'Indo-pacifique du CÉRIUM, Montréal, mars 2024.}

{\textbf{C.2 Rivest, Jozef}. \say{Vers un renouveau de la culture stratégique du Japon?} Colloque étudiant du Centre d'études et de recherche internationale de l'Université de Montréal (CÉRIUM), Montréal, mars 2023.} \par

{\textbf{C.1} Dufresne, Yannick, Axel Dery, \textbf{Jozef Rivest}, Catherine Ouellet and Etienne Gagnon. \say{What Social-Class Markers and Lifestyles Tell Us About the Left-Right Divide in Canada and Japan} Southern Political Science Association, St. Pete Beach, janvier 2023.}

% \subsection*{Submitted}

% \subsection*{Review and Resubmit}

\subsection*{Publications non évaluées par les pairs}

{\textbf{M.8 Rivest, Jozef}. 2024. \say{Le Japon plongé dans l'incertitude politique} \textit{Le Devoir}, 1er novembre 2024.}

{\textbf{M.7 Rivest, Jozef}. 2024. \say{Élections à Taïwan, une démocratie sous tension} \textit{Le Devoir}, 17 janvier 2024.} \par

{\textbf{M.6 Rivest, Jozef}. 2023. \say{Pourquoi la Chine est-elle si préoccupée par Taïwan?} \textit{Le Devoir}, 16 août 2023.} \par

{\textbf{M.5 Rivest, Jozef}. 2022. \say{Le Viêtnam : grand bénéficiaire de la rivalité sino-américaine} \textit{Blogue sur l'Asie du Sud-Est}, 4 janvier 2023.} \par 

{\textbf{M.4 Rivest, Jozef}. 2022. \say{Le panoptique technologique au service de l'autoritarisme} \textit{Blogue sur l'Asie du Sud-Est}, 30 novembre 2022.} \par

{\textbf{M.3 Rivest, Jozef}. \say{Le Parti communiste chinois ou le Parti de Xi Jinping? Retour sur le 20e Congrès national du PCC} \textit{La revue du CAIUM}, 1er novembre 2022.} \par

{\textbf{M.2 Rivest, Jozef}. 2022. \say{Guerre ou statu quo? Le cas de la mer de Chine méridionale} \textit{Blogue sur l'Asie du Sud-Est}, 30 novembre 2022.} \par 

{\textbf{M.1 Rivest, Jozef}. 2022. \say{Le Japon et la Covid-19 : paradoxe et culture}, \textit{L'Asie en 1000 mots}, 3 octobre 2022.} \par

\subsection*{En cours}

\subsubsection*{Logiciel}

{\textbf{Rivest, Jozef}. \say{factortable: Tableaux flexibles et intuitives pour présenter les résultats d'analyse factorielle dans un article.} \textbf{Package R}}

\subsubsection*{Article}

{\textbf{Rivest, Jozef}. \say{Pacifisme sous examen : Mesurer les normes japonaises au milieu de dynamiques géopolitiques changeantes.}}

{\textbf{Rivest, Jozef}, Mathieu Turgeon, Tetsuya Matsubayashi, Takeshi Iida, Yannick Dufresne. \say{Permanent ou temporaire ? Comprendre les préférences des Japonais envers l'immigration.}}

{Girard, Tyler, \textbf{Jozef Rivest}, Mathieu Turgeon, Yannick Dufresne, Takeshi Iida and Tetsuya Matsubayashi. \say{Soutien public au développement de technologies militaires émergentes dans les puissances « moyennes ».}} \par 

%--copy and paste this region  if you need more--
\end{rSection}

%--------------------------------------------------------------------------------
%    ACTIVITIES
%-----------------------------------------------------------------------------------------------
\begin{rSection}{Bourses et financement}
%--copy and paste this region  if you need more--
\begin{itemize} 
  \item Bourse doctorale, CRSH. 2025-2027. 120 000 CA\$
  \item Bourse d'excellence, Bourse Nissan. 2024. 3 000 CA\$
  \item Subventions de conférence, Centre pour l'étude de la citoyenneté démocratique. 2024. 1 000 CA\$
  \item Bourses de fin d'études, Université de Montréal, 2024. 4 000 CA\$
  \item Subventions de formation en méthodes, Centre pour l'étude de la citoyenneté démocratique. 2024. 3 000 CA\$
  \item Subvention de recherche, Centre pour l'étude de la citoyenneté démocratique. 2024. 2 500 CA\$
  \item Bourse d'excellence, Bourse Nissan. 2023. 3 000 CA\$
  \item Bourse de maîtrise, CRSH. 2022. 17 500 CA\$
  \item Bourse Desjardins, Desjardins. 2021. 1 000 CA\$
\end{itemize}

\end{rSection}


%% Affiliations académiques
\begin{rSection}{Affiliation} 
  {\em Association japonaise d'études électorales}\begin{CJK}{UTF8}{min} [日本選挙学会]\end{CJK} \hfill{\em 2024 - } \par
{\em Association canadienne de science politique} \hfill{\em 2024 - } \par
{\em Membre étudiant au Centre pour l'étude de la citoyenneté démocratique} \hfill{\em 2023 - } \par
{\em Chaire de leadership en enseignement des sciences sociales numériques} \hfill{\em 2022 - }

\end{rSection}
\clearpage
%----------------------------------------------------------------------------------------
%	SKILLS SECTION
%----------------------------------------------------------------------------------------
\begin{rSection}{Compétences}
{\bf Langues}
\begin{itemize}
    \item Français (langue maternelle)
    \item Anglais (C1)\footnote{Résultats TOEFL Internet : total 111, lecture 26, écoute 27, expression orale 30, écriture 28}
    \item Japonais (A2+, JLPT N4)
\end{itemize}
{\bf Langages de programmation}
\begin{itemize}
    \item \Rlogo 
    \item \LaTeX 
    \item Qualtrics
    \item Shiny
    \item Quarto/Markdown
\end{itemize}
\end{rSection}


\end{document}
